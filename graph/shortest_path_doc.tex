\documentclass{jsarticle}
\usepackage{amsmath,amssymb,url}
\usepackage[all]{xy}

\newcommand{\abs}[1]{\lvert #1 \rvert}
\newcommand{\R}{\mathbb{R}}

\everymath{\displaystyle}

\title{Shortest Path Replacement}
\author{hos}

\begin{document}
\maketitle


\section{問題}
無向・\textbf{正}重み付きグラフ $G = (V, E, w)$ ($w\colon E \to \R_{>0}$) および
$s, t \in V$ が与えられる.
各 $e \in E$ に対し,$G-e$ での $s$-$t$ ウォークの重みの最小値 $z_e$ を求めたい.

重みの加算・比較が可能として,
$O((\abs{V} + \abs{E}) \log(\abs{V}))$ 時間 $O(\abs{V} + \abs{E})$ 空間.
辺重みに $0$ を許す場合は $0$ を $\varepsilon$ にして解く.


\section{準備}
最短と言ったら重み最小のこととする.
$u, v \in V$ に対し,$G$ での $u$-$v$ 最短ウォークの重みを $d(u, v)$ と書く.

$d(s, t) = \infty$ のときは任意の $e \in E$ に対し $z_e = \infty$ である.
以下 $d(s, t) < \infty$ を仮定する.

$s = t$ のときは任意の $e \in E$ に対し $z_e = 0$ である.
以下 $s \ne t$ を仮定する.

$G$ での $s$-$t$ 最短単純パス $P$ を $1$ つとって固定する.
\begin{itemize}
  \item $s$ を始点とする最短路木 $S$ であって $P$ を含むものを $1$ つとって固定する.
      $S$ の頂点 $u$ に対し,$S$ での $s$-$u$ 単純パスを $S(u)$ と書く.
  \item $t$ を終点とする最短路木 $T$ であって $P$ を含むものを $1$ つとって固定する.
      $T$ の頂点 $u$ に対し,$T$ での $u$-$t$ 単純パスを $T(u)$ と書く.
\end{itemize}

\

\noindent\textbf{補題.}
$e \in E$ とし,
$G-e$ に $s$-$t$ ウォークが存在するとする.
このとき,$G-e$ での $s$-$t$ 最短ウォーク $Q$ であって,
$Q$ 上の任意の頂点 $u$ に対し,以下の少なくとも一方が成り立つようなものが存在する:
\begin{itemize}
  \item $Q$ の $u$ までの prefix が $S(u)$ である
  \item $Q$ の $u$ からの suffix が $T(u)$ である
\end{itemize}

\noindent\textbf{証明.}
条件をいずれも満たさない頂点の個数が最小となるような $Q$ をとる.
$Q$ 上のある頂点 $u$ が条件をいずれも満たさないと仮定して矛盾を導く.
\begin{itemize}
  \item $S(u)$ が $e$ を含まないとすると,$Q$ の prefix を $S(u)$ に置き換えて
      より良いウォークを得られるので矛盾.
  \item $T(u)$ が $e$ を含まないとすると,$Q$ の suffix を $T(u)$ に置き換えて
      より良いウォークを得られるので矛盾.
\end{itemize}

よって $S(u), T(u)$ は $e$ を含むとしてよい.
これらは最短路木の単純パスなのでそれぞれ $e$ をちょうど $1$ 回含む.
$e$ の前後に分解して $S(u) = Q_0 e Q_1$, $T(u) = Q_2 e Q_3$ とおく.

$S(u), T(u)$ が $e$ を同じ向きに通る場合.
$s$-$u$ ウォーク $Q_0 \overline{Q_2}$ および
$u$-$t$ ウォーク $\overline{Q_1} Q_3$ を考えることにより,
\begin{align*}
  w(Q_0) + w(e) + w(Q_1) &= d(s, u) \le w(Q_0) + w(Q_2) \\
  w(Q_2) + w(e) + w(Q_3) &= d(u, t) \le w(Q_1) + w(Q_3)
\end{align*}
を得るが,
両辺足して $w(e) \le 0$ となり矛盾.

$S(u), T(u)$ が $e$ を逆向きに通る場合,
$G-e$ での $s$-$t$ ウォーク $Q_0 Q_3$ を考えると,
その重みは
\[
  w(Q_0) + w(Q_3) < d(s, u) + d(u, t) = w(Q)
\]
なので $Q$ の最短性に矛盾.
\hfill ■

\

\noindent\textbf{注意.}
辺重みに $0$ を許すと補題は成り立たない:
\[
  \xymatrix{
    *++[o][F]{s} \ar@{-}[rd]^1 \ar@{-}[rr]^1 & & *++[o][F]{u} \ar@{-}[ld]^0 \ar@{-}[rr]^1 & & *++[o][F]{t} \\
    & *++[o][F]{a} \ar@{--}[rr]^0_e & & *++[o][F]{b} \ar@{-}[lu]^0 \ar@{-}[ru]^1 &
  }
\]
上図で,$P = sabt$ として,最短路木として
\begin{itemize}
  \item $S$ の辺は $sa, ab, bt, bu$
  \item $T$ の辺は $sa, ab, bt, ua$
\end{itemize}
となるものをとると,
$Q = sut, saut, subt, saubt$ のどれについても,頂点 $u$ で条件を満たさない.

有向グラフの場合も補題は成り立たない:
\[
  \xymatrix{
    *++[o][F]{s} \ar[rd]^1 \ar[rr]^{10} & & *++[o][F]{u} \ar[ld]^1 \ar[rr]^{10} & & *++[o][F]{t} \\
    & *++[o][F]{a} \ar@{-->}[rr]^1_e & & *++[o][F]{b} \ar[lu]^1 \ar[ru]^1 &
  }
\]
上図で $G-e$ での $s$-$t$ ウォークは $Q = sut$ のみだが,
$d(s, u) = d(u, t) = 3$ より頂点 $u$ で条件を満たさない.


\section{解法}
$P = (s = p_0, e_0, p_1, e_1, \ldots, e_{k-1}, p_k = t)$
($p_i \in V$, $e_i \in E$, $e_i$ の両端点は $p_i, p_{i+1}$)
とおく.

$e \in E \setminus \{e_0, \ldots, e_{k-1}\}$ に対しては
$z_e = d(s, t)$ である.

$u \in V$ に対し,$l(u), r(u)$ を以下で定める:
\begin{itemize}
  \item $u$ が $S$ の頂点のとき,$P$ と $S(u)$ の共通部分 (prefix である) を
      $(p_0, e_0, \ldots, e_{l(u)-1}, p_{l(u)})$ とする.
      そうでないとき $l(u) = +\infty$ とする.
  \item $u$ が $T$ の頂点のとき,$P$ と $T(u)$ の共通部分 (suffix である) を
      $(p_{r(u)}, e_{r(u)}, \ldots, e_{k-1}, p_k)$ とする.
      そうでないとき $r(u) = -\infty$ とする.
\end{itemize}

このとき,
\[
  z_{e_i}
  = \min\{ \quad
      d(s, u) + w(e) + d(v, t)
      \quad \mid \quad
      e \in E \setminus \{e_i\},\quad \text{$e$ の両端点は $u, v$},\quad l(u) \le i < r(v)
  \quad \}
\]
が成り立つ.

\

\noindent\textbf{$\le$ の証明.}
右辺の条件を満たす $e, u, v$ を任意にとる.
\begin{itemize}
  \item $l(u)$ の定め方と $l(u) \le i$ より,$S(u)$ は $e_i$ を含まない.
  \item $r(v)$ の定め方と $i <   r(v)$ より,$T(v)$ は $e_i$ を含まない.
\end{itemize}

すると,$S(u) e T(v)$ が $G-e_i$ のウォークとなり,
\[
  z_{e_i} \le w(S(u) e T(v)) = d(s, u) + w(e) + d(v, t)
\]
である.

\noindent\textbf{$\ge$ の証明.}
$z_{e_i} = \infty$ のときはよい.
そうでないとき,
$G-e_i$ について補題を適用して,
$G-e_i$ での $s$-$t$ 最短ウォーク $Q$ を $1$ つとる.
$s \ne t$ および
\begin{itemize}
  \item $Q$ の $s$ までの prefix は $S(s)$ である
  \item $Q$ の $t$ からの suffix は $T(t)$ である
\end{itemize}
ということと補題の性質を合わせると,
$Q$ に含まれる辺 $e$ であって,両端点を通る順に $u, v$ として,
\begin{itemize}
  \item $Q$ の $u$ までの prefix は $S(u)$ である
  \item $Q$ の $v$ からの suffix は $T(v)$ である
\end{itemize}
ようなものがとれる.
このとき,$Q$ が $e_i$ を含まないことから
$e \in E \setminus \{e_i\}$ および $l(u) \le i < r(v)$ が成り立ち,
\[
  z_{e_i} = w(Q) = d(s, u) + w(e) + d(v, t)
\]
であるからよい.
\hfill ■


\end{document}
