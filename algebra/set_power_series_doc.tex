\documentclass{jsarticle}
\usepackage{amsmath,amssymb}
\usepackage[all]{xy}

\newcommand{\abs}[1]{\lvert #1 \rvert}
\newcommand{\minuseq}{\mathrel{-}=}
\newcommand{\pluseq}{\mathrel{+}=}
\renewcommand{\labelitemii}{$\circ$}

\everymath{\displaystyle}

\title{集合冪級数}
\author{hos}

\begin{document}
\maketitle


係数環を $R$ とする.
$x_0, \ldots, x_{n-1}$ を不定元として,
$I \subseteq [n] := \{0, \ldots, n-1\}$ に対し
$x^I = \prod_{i\in I} x_i$ と書く.
$I$ と $\sum_{i\in I} 2^i$ をしばしば同一視する.

\section{subset convolution}
subset convolution とは,
$R 2^X := R[x_0, \ldots, x_{n-1}] / (x_0^2, \ldots, x_{n-1}^2)$ での積.

不定元 $t$ を導入して,
$R[x_0, \ldots, x_{n-1}] / (x_0(x_0-t), \ldots, x_{n-1}(x_{n-1}-t))$ での積
を計算して $t \to 0$ とすることにする.
中国剰余定理より,これは $\{0, t\}^n$ で多点評価して各点積をとって補間すればよい.
各次元では $(a, b) \mapsto (a, a+bt)$, $(a, b) \mapsto (a, (b-a)/t)$ という変換になるので,
$t$ で割るのを後回しにすると考えて,
以下のアルゴリズムが得られる:
\begin{enumerate}
  \item 入力のそれぞれについて,$x^I$ を $x^I t^{\abs{I}}$ に置き換える.
  \item 入力のそれぞれについて,累積和をとる:$(a_I)_I \mapsto \left(\sum_{J\subseteq I} a_J\right)_I$
  \item $t$ の多項式として各点積をとる.
  \item 差分をとる:$(a_I)_I \mapsto \left(\sum_{J\subseteq I} (-1)^{\abs{I}-\abs{J}} c_J\right)_I$
  \item $[x^I t^{\abs{I}}]$ をとると出力の $[x^I]$ である.
\end{enumerate}

2, 4 がボトルネックで $O(2^n n^2)$ 時間.
3 は FFT で $O(2^n n \log(n))$ 時間にもできるが恩恵が少ない.

\

メモリアクセスを考慮して,ステップ 2, 3, 4 を再帰で実装する (segment tree 上の DFS).

入力を $A, B \in R 2^X$ とする.
$2^n \times (n+1)$ 配列 $a, b$ を用意し,
$0 \le h < 2^n$, $0 \le k \le n$ について,
\begin{itemize}
  \item $a[h][k] \gets [\abs{h} = k] \cdot [x^h] A$
  \item $b[h][k] \gets [\abs{h} = k] \cdot [x^h] B$
\end{itemize}
として,
以下の $\mathtt{rec}(n, 0)$ を呼ぶ.
すると出力が $[x^h] A(x) B(x) = a[h][\abs{h}]$ として得られる.
\begin{quote}
  $\mathtt{rec}(m, h_0)$ ($0 \le m \le n$, $0 \le h_0 < 2^n$, $2^m \mid h_0$)
  \begin{itemize}
    \item $m > 0$ のとき
      \begin{enumerate}
        \item $h_0 \le h < h_0 + 2^{m-1}$ について,
          \textcircled{1}各 $k$ について,
          \begin{itemize}
            \item $a[h + 2^{m-1}][k] \pluseq a[h][k]$
            \item $b[h + 2^{m-1}][k] \pluseq b[h][k]$
          \end{itemize}
          とする.
        \item $\mathtt{rec}(m-1, h)$, $\mathtt{rec}(m-1, h + 2^{m-1})$ を呼ぶ.
        \item $h_0 \le h < h_0 + 2^{m-1}$ について,
          \textcircled{2}各 $k$ について,
          $a[h + 2^{m-1}][k] \minuseq a[h][k]$ とする.
      \end{enumerate}
    \item $m = 0$ のとき
      \begin{enumerate}
      \item \textcircled{3}各 $k$ について,
        $a[h_0][k] \gets \sum_{0\le l\le k} a[h_0][l] \cdot b[h_0][k-l]$ とする.
      \end{enumerate}
  \end{itemize}
\end{quote}

\textcircled{1}, \textcircled{2}, \textcircled{3} で操作する $k, l$ の範囲は
$0 \le k \le n$, $0 \le l \le k$ より狭くできる.

\textcircled{1} について,
非 $0$ の値が入る場所を考えると,
$\abs{h} - \abs{h_0} \le k \le \abs{h}$ としてよい.
例えば $n = 3$ では以下の表のようになる.

\begin{center}
  \begin{tabular}{|l|c|c|c|c|}\hline
    \multicolumn{5}{|c|}{$m=3$} \\\hline
        & $0$ & $1$ & $2$ & $3$ \\\hline
    $0$ & $*$ &     &     &     \\\hline
    $1$ &     & $*$ &     &     \\\hline
    $2$ &     & $*$ &     &     \\\hline
    $3$ &     &     & $*$ &     \\\hline
    $4$ &     & $*$ &     &     \\\hline
    $5$ &     &     & $*$ &     \\\hline
    $6$ &     &     & $*$ &     \\\hline
    $7$ &     &     &     & $*$ \\\hline
  \end{tabular}
  →\
  \begin{tabular}{|l|c|c|c|c|}\hline
    \multicolumn{5}{|c|}{$m=2$} \\\hline
        & $0$ & $1$ & $2$ & $3$ \\\hline
    $0$ & $*$ &     &     &     \\\hline
    $1$ &     & $*$ &     &     \\\hline
    $2$ &     & $*$ &     &     \\\hline
    $3$ &     &     & $*$ &     \\\hline
    $4$ & $*$ & $*$ &     &     \\\hline
    $5$ &     & $*$ & $*$ &     \\\hline
    $6$ &     & $*$ & $*$ &     \\\hline
    $7$ &     &     & $*$ & $*$ \\\hline
  \end{tabular}
  →\
  \begin{tabular}{|l|c|c|c|c|}\hline
    \multicolumn{5}{|c|}{$m=1$} \\\hline
        & $0$ & $1$ & $2$ & $3$ \\\hline
    $0$ & $*$ &     &     &     \\\hline
    $1$ &     & $*$ &     &     \\\hline
    $2$ & $*$ & $*$ &     &     \\\hline
    $3$ &     & $*$ & $*$ &     \\\hline
    $4$ & $*$ & $*$ &     &     \\\hline
    $5$ &     & $*$ & $*$ &     \\\hline
    $6$ & $*$ & $*$ & $*$ &     \\\hline
    $7$ &     & $*$ & $*$ & $*$ \\\hline
  \end{tabular}
  →\
  \begin{tabular}{|l|c|c|c|c|}\hline
    \multicolumn{5}{|c|}{$m=0$} \\\hline
        & $0$ & $1$ & $2$ & $3$ \\\hline
    $0$ & $*$ &     &     &     \\\hline
    $1$ & $*$ & $*$ &     &     \\\hline
    $2$ & $*$ & $*$ &     &     \\\hline
    $3$ & $*$ & $*$ & $*$ &     \\\hline
    $4$ & $*$ & $*$ &     &     \\\hline
    $5$ & $*$ & $*$ & $*$ &     \\\hline
    $6$ & $*$ & $*$ & $*$ &     \\\hline
    $7$ & $*$ & $*$ & $*$ & $*$ \\\hline
  \end{tabular}
\end{center}

\textcircled{2} について,
出力に寄与する場所を考える (転置を考える) と,
$\abs{h} \le k \le \abs{h} + (n - (m-1) - \abs{h_0})$ としてよい.
例えば $n = 3$ では以下の表のようになる.

\begin{center}
  \begin{tabular}{|l|c|c|c|c|}\hline
    \multicolumn{5}{|c|}{$m=3$} \\\hline
        & $0$ & $1$ & $2$ & $3$ \\\hline
    $0$ & $*$ &     &     &     \\\hline
    $1$ &     & $*$ &     &     \\\hline
    $2$ &     & $*$ &     &     \\\hline
    $3$ &     &     & $*$ &     \\\hline
    $4$ &     & $*$ &     &     \\\hline
    $5$ &     &     & $*$ &     \\\hline
    $6$ &     &     & $*$ &     \\\hline
    $7$ &     &     &     & $*$ \\\hline
  \end{tabular}
  ←\
  \begin{tabular}{|l|c|c|c|c|}\hline
    \multicolumn{5}{|c|}{$m=2$} \\\hline
        & $0$ & $1$ & $2$ & $3$ \\\hline
    $0$ & $*$ & $*$ &     &     \\\hline
    $1$ &     & $*$ & $*$ &     \\\hline
    $2$ &     & $*$ & $*$ &     \\\hline
    $3$ &     &     & $*$ & $*$ \\\hline
    $4$ &     & $*$ &     &     \\\hline
    $5$ &     &     & $*$ &     \\\hline
    $6$ &     &     & $*$ &     \\\hline
    $7$ &     &     &     & $*$ \\\hline
  \end{tabular}
  ←\
  \begin{tabular}{|l|c|c|c|c|}\hline
    \multicolumn{5}{|c|}{$m=1$} \\\hline
        & $0$ & $1$ & $2$ & $3$ \\\hline
    $0$ & $*$ & $*$ & $*$ &     \\\hline
    $1$ &     & $*$ & $*$ & $*$ \\\hline
    $2$ &     & $*$ & $*$ &     \\\hline
    $3$ &     &     & $*$ & $*$ \\\hline
    $4$ &     & $*$ & $*$ &     \\\hline
    $5$ &     &     & $*$ & $*$ \\\hline
    $6$ &     &     & $*$ &     \\\hline
    $7$ &     &     &     & $*$ \\\hline
  \end{tabular}
  ←\
  \begin{tabular}{|l|c|c|c|c|}\hline
    \multicolumn{5}{|c|}{$m=0$} \\\hline
        & $0$ & $1$ & $2$ & $3$ \\\hline
    $0$ & $*$ & $*$ & $*$ & $*$ \\\hline
    $1$ &     & $*$ & $*$ & $*$ \\\hline
    $2$ &     & $*$ & $*$ & $*$ \\\hline
    $3$ &     &     & $*$ & $*$ \\\hline
    $4$ &     & $*$ & $*$ & $*$ \\\hline
    $5$ &     &     & $*$ & $*$ \\\hline
    $6$ &     &     & $*$ & $*$ \\\hline
    $7$ &     &     &     & $*$ \\\hline
  \end{tabular}
\end{center}

\textcircled{3} について,
非 $0$ の値が入る場所を考えて $k \le 2\abs{h_0}$ としてよく
($2$ 個の多項式の積であることを用いた),
出力に寄与する場所を考えて $\abs{h_0} \le k$ としてよい.
非 $0$ の値が入る場所を考えて $0 \le l \le \abs{h_0}$, $0 \le k-l \le \abs{h_0}$ としてよい.

さらに,\textcircled{2} について,
非 $0$ の値が入る場所を考えて $k \le 2\abs{h}$ としてよい.


\section{exp}
$[x^\emptyset] A = 0$ なる $A \in R 2^X$ に対して,
$\exp(A) := \sum_{i=0}^\infty \frac{A^i}{i!} = \sum_{i=0}^n \frac{A^i}{i!}$ を求めたい.
$\frac{A^i}{i!}$ は環演算のみで定義できることに注意する.

$x_{n-1}$ の次数で分けて
$A = A_0 + A_1 x_{n-1}$ とおくと
\[
  \exp(A)
  = \exp(A_0) \exp(A_1 x_{n-1})
  = \exp(A_0) (1 + A_1 x_{n-1})
  = \exp(A_0) + \exp(A_0) A_1 x_{n-1}
\]
となり,サイズ $n-1$ の subset convolution $1$ 回とサイズ $n-1$ の $\exp$ に帰着できる.
$O(2^n n^2)$ 時間.

\

毎回補間をせず,多点評価した状態で持っておくことができる.

$I \subseteq [n-1]$ とする.
$\exp(A_0), \exp(A_1)$ を $x_i = [i \in I] \cdot t$ で評価した値を
それぞれ $a_0, a_1 \in R[t]$ とすると,
$\exp(A)$ をさらに $x_{n-1} = 0, t$ で評価した値はそれぞれ $a_0, a_0 + a_0 a_1 t$ となる.
ここで補間時に $x_{n-1}$ の軸から差分をとると,
$(a_0, a_0 + a_0 a_1 t) \mapsto (a_0, a_0 a_1)$ となるが,
$a_0$ が $O(t^n)$ ずれていても,残り $n-1$ 軸の変換後 $t \to 0$ とすると消えるので,
出力に影響がないことがわかる.

以上より,サイズ $n$ の部分問題としては $2^n \times (n+1)$ 配列を求めればよい.

\

実測だと毎回 subset convolution を呼ぶほうが速い.
TODO: 添え字の範囲を詰められていないかもしれないし,本当に枝刈りが効きにくくなっているかもしれない.


\section{合成}
EGF $f(y) = \sum_{i=0}^\infty f_i \frac{y^i}{i!}$ ($f_i \in R$) と
$[x^\emptyset] A = 0$ なる $A \in R 2^X$ に対して,
$f(A) = \sum_{i=0}^\infty f_i \frac{A^i}{i!} = \sum_{i=0}^n f_i \frac{A^i}{i!}$ を求めたい.

$x_{n-1}$ の次数で分けて
$A = A_0 + A_1 x_{n-1}$ とおくと
\[
  f(A) = f(A_0) + f'(A_0) A_1 x_{n-1}
\]
であるから,
$f$ の $i$ 階微分を $f^{(i)}$ と書いて,
サイズ $m$ で $f^{(0)}, \ldots, f^{(n-m)}$ との合成を求める問題に帰着される.
base case は $f_0, \ldots, f_n$ である.
時間計算量は $\sum_{0\le m<n} (n-m) \cdot O(2^m m^2) = O(2^n n^2)$.

\

以下のように subset convolution を除いてサイズ $2^n$ の配列上で実装できる:
\[
  \xymatrix{
    & [0, 1) & [1, 2) & [2, 4) & [4, 8) \\
    f_3 \ar[r] & f^{(3)}(A[0,1)) \ar[dr]_{A[1,2)} & & & \\
    & & f^{(2)}(A[0,2)) \ar[ddr]_(.7){A[2,4)} & & \\
    f_2 \ar[r] & f^{(2)}(A[0,1)) \ar[ddr]_{A[1,2)} \ar[drr] & & & \\
    & & & f^{(1)}(A[0,4)) \ar[dddr]_(.7){A[4,8)} & \\
    & & f^{(1)}(A[0,2)) \ar[dddr]_(.7){A[2,4)} \ar[ddrr] & & \\
    f_1 \ar[r] & f^{(1)}(A[0,1)) \ar[dddr]_{A[1,2)} \ar[ddrr] \ar[drrr] & & & \\
    & & & & f^{(0)}(A[4,8)) \ar[dddd] & \\
    & & & f^{(0)}(A[0,4)) \ar[ddd] & \\
    & & f^{(0)}(A[0,2)) \ar[dd] & & \\
    f_0 \ar[r] & f^{(0)}(A[0,1)) \ar[d] & & & \\
    & f(A)[0,1) & f(A)[1,2) & f(A)[2,4) & f(A)[4,8) \\
  }
\]

\

(TODO: 多点評価した状態で持つ方針で定数倍を詰める)

\

多項式 $f(y) = \sum_i f_i y^i \in R[y]$ と
$A \in R 2^X$ に対しても
$f(A)$ が定まる.
これは,定数項 $a = [x^\emptyset] A$ を分けて Taylor 展開して,
\[
  f(A) = f(a + (A-a)) = \sum_{i=0}^\infty f^{(i)}(a) \frac{(A-a)^i}{i!}
\]
によって EGF の場合に帰着できる.


\section{転置}
$A \in R 2^X$ を固定するとき,
$A$ 倍写像 $R 2^X \to R 2^X$ の転置は,
$A$・入力・出力すべてを reverse しての subset convolution となる.
これは,
$A$ 倍写像を行列で書くと $([i \supseteq j] \cdot a_{i-j})_{i,j}$ となり,
その転置行列が
\[
  ([j \supseteq i] \cdot a_{j-i})_{i,j}
  = \bigl(\bigl[[n]-i \supseteq [n]-j\bigr] \cdot a_{([n]-i)-([n]-j)}\bigr)_{i,j}
\]
となることからわかる.

\

$[x^\emptyset] B = 0$ なる
$B \in R 2^X$ を固定するとき,
EGF 合成 $[n+1] \to R 2^X; f \mapsto f(B)$ の転置は,
適切に reverse を挟むことで,
EGF power projection $R 2^X \to [n+1]; A \mapsto \left([x^{[n]}] A \frac{B^i}{i!}\right)_i$
である.

直接アルゴリズムを導出する.
不定元 $t$ を導入して,
答えの EGF $[x^{[n]}] \sum_{i=0}^\infty A \frac{(tB)^i}{i!} = [x^{[n]}] A \exp(tB)$ を求めたい.
$x_{n-1}$ の次数で分けて
$A = A_0 + A_1 x_{n-1}$,
$B = B_0 + B_1 x_{n-1}$ とおくと,
\begin{align*}
  [x^{[n]}] A \exp(tB)
  &= [x^{[n]}] (A_0 + A_1 x_{n-1}) \exp(t (B_0 + B_1 x_{n-1})) \\
  &= [x^{[n]}] (A_0 + A_1 x_{n-1}) \exp(t B_0) (1 + t B_1 x_{n-1})) \\
  &= [x^{[n]}] \left(A_0 \exp(t B_0) + (A_1 + t A_0 B_1) \exp(t B_0) x_{n-1}\right) \\
  &= [x^{[n-1]}] (A_1 + t A_0 B_1) \exp(t B_0)
\end{align*}
であるから,
サイズ $m$ では $t$ の $n-m$ 次式が登場する (power projection を $n-m+1$ 回解けばよい).
base case では $\exp(t B) = 1$ である.

サイズ $m$ の部分問題を $[x^{[m]}] \left( C_m \frac{B^i}{i!} \right)_i$
($C_m \in R 2^{\{x_0,\ldots,x_{m-1}\}} [t]$) とおくと,
計算過程は以下のようになる:
\[
  \xymatrix{
    [0, 4) & [4, 6) & [6, 7) & [7, 8) & \\
    & & & [t^3] C_0 \ar[r] & \mathit{out}_3 \\
    & & [t^2] C_1 \ar[ur]_{B[1,2)} & & \\
    & & & [t^2] C_0 \ar[r] & \mathit{out}_2 \\
    & [t^1] C_2 \ar[uur]_(.3){B[2,4)} \ar[urr] & & & \\
    & & [t^1] C_1 \ar[uur]_{B[1,2)} & & \\
    & & & [t^1] C_0 \ar[r] & \mathit{out}_1 \\
    [t^0] C_3 \ar[uuur]_(.3){B[4,8)} \ar[uurr] \ar[urrr] & & & & \\
    & [t^0] C_2 \ar[uuur]_(.3){B[2,4)} \ar[uurr] & & &\\
    & & [t^0] C_1 \ar[uuur]_{B[1,2)} & & \\
    & & & [t^0] C_0 \ar[r] & \mathit{out}_0 \\
    A[0,4) \ar[uuuu] & A[4,6) \ar[uuu] & A[6,7) \ar[uu] & A[7,8) \ar[u] & \\
  }
\]

\

$B \in R 2^X$ を固定するとき,
多項式合成 $R[y] \to R 2^X; f \mapsto f(B)$ の転置は,
適切に reverse を挟むことで,
power projection $A \mapsto \left([x^{[n]}] A B^i\right)_i$
である.

\

定数項 $b = [x^\emptyset] B$ を分けて,
\[
  A B^i
  = A (b + (B-b))^i
  = \sum_j \frac{i!}{(i-j)!} b^{i-j} A \frac{(B-b)^j}{j!}
\]
となるので EGF の場合に帰着できる.


\end{document}
