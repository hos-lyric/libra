\documentclass{jsarticle}
\usepackage{amsmath,amssymb,amsthm,ascmac,enumerate,framed}

\everymath{\displaystyle}
\DeclareMathOperator{\Mod}{mod}
\DeclareMathOperator{\Res}{Res}
\DeclareMathOperator{\ord}{ord}
\newcommand{\N}{\mathbb{N}}
\newcommand{\Q}{\mathbb{Q}}
\newcommand{\Z}{\mathbb{Z}}
\newcommand{\longto}{\longrightarrow}
\newcommand{\middlemid}{\mathrel{}\middle|\mathrel{}}

\theoremstyle{definition}
\newtheorem*{Dfn}{定義}
\newtheorem*{Exm}{例}
\newtheorem{Prp}{命題}
\renewcommand{\proofname}{\textbf{証明}}
\newenvironment{dfn}{\vspace{1ex}\begin{screen}\begin{Dfn}}{\end{Dfn}\end{screen}\vspace{1ex}}
\newenvironment{exm}{\begin{leftbar}\begin{Exm}}{\end{Exm}\end{leftbar}}
\newenvironment{prp}{\vspace{1ex}\begin{screen}\begin{Prp}}{\end{Prp}\end{screen}}
\newenvironment{prf}{\begin{leftbar}\begin{proof}}{\end{proof}\end{leftbar}}

\title{メモ (形式的冪級数) (書きかけ)}
\author{hos}

\begin{document}
\maketitle

大事なこと:
収束のことは考えない.
その代わり,知らない演算をしない.

$\N$ は非負整数全体の集合とする\footnote{普段は $\Z_{\ge 0}$ とかを使って $\N$ という記号を避けようと思っているが,$\sum$ の下にたくさん書くので仕方なく.}.
環と言ったら乗法の単位元の存在を仮定する.

\section{形式的冪級数環}
以降,$R$ を可換環とする.

$X$ を不定元として,
$R$ の加算個の直積 $\prod_{i\in\N} R$ の元 $(a_i)_{i\in\N}$ を形式的に
$\sum_{i\in\N} a_i X^i$ (あるいは $a_0 + a_1 X + a_2 X^2 + \cdots$) と書いたものの集合を $R[[X]]$ とする.
この元を $a(X)$ のように書くこともある\footnote{不定元を書かず単に $a$ のように書くのも綺麗だが,今回は積と合成が混同しないことを重視.}.
$a_i$ を $a(X)$ の $i$ 次の\textbf{係数} (あるいは $X^i$ の係数) と呼び,$[X^i] a(X)$ のように書く.
$0$ 次の係数を\textbf{定数項}と呼ぶ.

$\sum_{i\in\N} a_i X^i, \sum_{i\in\N} b_iX^i \in R[[X]]$ に対し,加法と乗法を
\begin{align*}
  \sum_{i\in\N} a_i X^i + \sum_{i\in\N} b_i X^i &:= \sum_{i\in\N} (a_i + b_i) X^i, \\
  \left(\sum_{i\in\N} a_i X^i\right) \left(\sum_{i\in\N} b_i X^i\right) &:= \sum_{k\in\N} \left(\sum_{i,j\in\N,\,i+j=k} a_i b_j\right) X^k
\end{align*}
で定めると,可換環になることを示す.
加法の単位元は $0 = 0 + 0 X + 0 X^2 + \cdots$,
乗法の単位元は $1 = 1 + 0 X + 0 X^2 + \cdots$.
乗法の結合法則のみやや非自明で,
$\left(\left(\sum_{i\in\N} a_i X^i\right) \left(\sum_{i\in\N} b_i X^i\right)\right) \left(\sum_{i\in\N} c_i X^i\right)$ の
$m$ 次の係数が
\[
  \sum_{l,k\in\N,\,l+k=m} \left(\sum_{i,j\in\N,\,i+j=k} a_i b_j\right) c_k = \sum_{i,j,k\in\N,\,i+j+k=m} a_i b_j c_k
\]
となることから従う.
この可換環 $R[[X]]$ を,\textbf{$R$ 係数形式的冪級数環}と呼ぶ.

さらに,$r \in R$ はそのまま $r + 0 X + 0 X^2 + \cdots \in R[[X]]$ とみれるので,
包含 $R \lhook\joinrel\longrightarrow R[[X]]$ により $R[[X]]$ は $R$ 代数でもある.

\begin{exm}
  $(1 + X) (1 + X + X^2 + X^3 + \cdots) = 1 + 2 X + 2 X^2 + 2 X^3 + \cdots$.
\end{exm}

$a(X) \in R[[X]]$ に対し,
集合 $a(X) R[[X]] := \{ a(X) b(X) \mid b(X) \in R[[x]] \} \subseteq R[[X]]$ は
$R[[X]]$ のイデアルである.
$b(X), c(X) \in R[[X]]$ が $b(X) - c(X) \in a(X) R[[X]]$ を満たすことを
$b(X) \equiv c(X) \pmod{a(X)}$ と書く.
$n \in \N$ に対し,$\Mod X^n$ での合同は,$n$ 次未満の係数が等しいことを表す.

\begin{exm}
  $0 + 1 X + 2 X^2 + 3 X^3 + 4 X^4 + \cdots \equiv X + 2 X^2 \pmod{X^3}$.
\end{exm}

次の命題は,突き詰めると環の位相の話になるが,本稿では技術的な補題として用いる.

\begin{prp}
  \label{prp:mod-limit}
  $a(X), b(X) \in R[[X]]$ について,
  $a(X) = b(X)$ である必要十分条件は,任意の $n \in \N$ に対して $a(X) \equiv b(X) \pmod{X^n}$ であること.
\end{prp}
\begin{prf}
  (必要性)
  明らか.

  (十分性)
  任意の $i \in \N$ に対し,$n = i + 1$ ととって $a(X) \equiv b(X) \pmod{X^{i+1}}$ なので,$a_i = b_i$ となる.
\end{prf}


\section{形式的 Laurent 級数環}
$S = \{ X^n \mid n \in \N \}$ は $R[[X]]$ の積閉集合であるから,局所化 $S^{-1} R[[X]]$ が考えられる.
これを $R((X))$ と書き,\textbf{$R$ 係数形式的 Laurent 級数環}という.
$S$ は零因子を含まないので自然な $R[[X]] \longto R((X))$ は単射であり,$R[[X]] \subseteq R((X))$ とみなせる.

局所化の構成を確認すれば,$R((X))$ の元は形式的に
$\frac{\sum_{i\in\N} a_i X^i}{X^n} = \sum_{i\in\N} a_i X^{i-n}$ と書いてよく,
\[
  R((X)) = \left\{ \sum_{i\in\Z} a_i X^i \middlemid \text{$a_i \ne 0$ なる $i \in \Z_{\le 0}$ は有限個} \right\}
  = \left\{ \sum_{i\in\Z,\,i\ge m} a_i X^i \middlemid m \in \Z \right\}
\]
である\footnote{以降,$\sum_{i\in\Z} a_i X^i \in R((X))$ と書いたら $a_i \ne 0$ なる $i \in \Z_{\le 0}$ は有限個であることも主張する.}.
環の演算は,$\sum_{i\in\Z,\,i\ge m} a_i X^i, \sum_{i\in\Z,\,i\ge n} b_i X^i \in R((X))$ に対し,
\begin{align*}
  \sum_{i\in\Z,\,i\ge m} a_i X^i + \sum_{i\in\Z,\,i\ge n} b_i X^i &= \sum_{i\in\Z,\,i\ge\min\{m,n\}} (a_i + b_i) X^i, \\
  \left(\sum_{i\in\Z,\,i\ge m} a_i X^i\right) \left(\sum_{i\in\Z,\,i\ge n} b_i X^i\right) &= \sum_{k\in\Z,\,k\ge m+n} \left(\sum_{i,j\in\Z,\,i\ge m,\,j\ge n,\,i+j=k} a_i b_j\right) X^k
\end{align*}
となる ($1$ 式目では $i < m$ のとき $a_i = 0$,$i < n$ のとき $b_i = 0$ とする.$2$ 式目では内側の $\sum$ が有限和となる).

$a(X) = \sum_{i\in\Z} a_i X^i \in R((X))$ に対し,$a_i \ne 0$ なる最小の $i \in \Z$ を $\ord(a(X))$ で表す.
ただし $\ord(0) = \infty$ とする.
$\ord(a(X))$ 次を\textbf{最低次}と呼ぶ.

$R$ が整域ならば,$R[[X]]$ や $R((X))$ も整域であり (最低次の係数に注目する),
$\ord\colon R((X)) \longto \Z \cup \{\infty\}$ は付値を与える.


\section{乗法の逆元}
% TODO: R((X)) について先にやる?
環の単元とは,乗法の逆元をもつ元のこと.可逆元.$1$ の約数.
$R$ の単元全体を $R^\times$ と書く.

\begin{prp}
  $a(X) = \sum_{i\in\N} a_i X^i \in R[[X]]$ について,
  $a(X) \in R[[X]]^\times \iff a_0 \in R^\times$.
\end{prp}
\begin{prf}
  ($\Longrightarrow$)
  $b(X) = \sum_{i\in\N} b_i X^i \in R[[X]]$ が $a(X) b(X) = 1$ を満たすとすると,
  定数項を比較して,$a_0 b_0 = 1$ である.

  ($\Longleftarrow$)
  $a_0 \in R^\times$ のとき,$a_0 r = 1$ なる $r \in R$ をとって,
  $b(X) = \sum_{i\in\N} b_i X^i \in R[[X]]$ を
  \begin{align*}
    b_0 &= r, \\
    b_i &= -r \sum_{j\in\N,\,1\le j\le i} a_j b_{i-j} \quad (i \ge 1)
  \end{align*}
  として定めると,$a(X) b(X) = 1$ を満たす.
\end{prf}

$a(X) \in R[[X]]$ の乗法の逆元が存在するとき,それは一意なので,
$a(X)^{-1}$ や $\frac{1}{a(X)}$ と書く.

\begin{exm}
  $r \in R$ に対し,$(1 - r X)^{-1} = \sum_{i\in\N} r^i X^i$.
\end{exm}

\begin{prp}
  $R$ を整域とする.
  $a(X) = \sum_{i\in\Z} a_i X^i \in R((X)) \setminus \{0\}$ について,
  $\ord(a(X)) = m$ とすると,
  $a(X) \in R((X))^\times \iff a_m \in R^\times$.
\end{prp}
\begin{prf}
  ($\Longrightarrow$)
  $b(X) = \sum_{i\in\Z} b_i X^i \in R((X)) \setminus \{0\}$ が $a(X) b(X) = 1$ を満たすとすると,
  $\ord(b(X)) = n$ として,
  $m + n = \ord(a(X) b(X)) = 0$ 次の係数を比較して $a_m b_n = 1$ である.

  ($\Longleftarrow$)
  $a(X) = X^m c(X)$, $c(X) \in R[[X]]$ と書けて,
  $[X^0] c(X) = a_m \in R^\times$ より
  $c(X)$ の $R[[X]]$ における乗法の逆元 $c(X)^{-1} \in R[[X]]$ がとれる.
  すると,$X^{-m} c(X)^{-1} \in R((X))$ であり,
  $a(X) (X^{-m} c(X)^{-1}) = 1$ となる.
\end{prf}

とくに,$R$ が体ならば,$R((X))$ も体である.

\begin{exm}
  $R$ が整域でないとき,
  $(\Z/6\Z)((X))$ において $(2 + 3X) (3X^{-1} + 2) = 1$,といった場合がある.
\end{exm}

ここまでで定めた加減乗除については,一般の $R$ 代数で成り立つことを用いて普通の計算ができるし,普通の表記をする.

\begin{exm}
  $a(X) \in R((X))$ に対して,$a(X)^2$ とは $a(X) a(X)$ のことであり,
  $a(X)^2$ の逆元は $(a(X)^{-1})^2$ であり $a(X)^{-2}$ と書く.
\end{exm}

\begin{exm}
  正の整数 $n$ に対し,$(1 - X)^{-n} = \sum_{i\in\N} \binom{i+n-1}{n-1} X^i$.
\end{exm}

\begin{exm}
  $\Q((X))$ において,$\frac{X}{X^2 + X^3} = X^{-1} - 1 + X - X^2 + X^3 - \cdots$.
\end{exm}


\section{合成}
形式的冪級数の合成は,$X$ の部分に「代入」していいものは定数項が $0$ でなければならない (すなわち,イデアル $X R[[X]]$ の元である) ことに注意を要する.

\begin{dfn}
  $a(X) = \sum_{i\in\N,\,i\ge 1} a_i X^i \in X R[[X]]$ と
  $b(X) = \sum_{i\in\N} b_i X^i \in R[[X]]$ に対し,
  $b(X)$ と $a(X)$ の\textbf{合成} $(b \circ a)(X)$ を
  \[
    % (a \circ b)(X) := \sum_{i\in\N} \left(\sum_{j\in\N,\, k_1,\ldots,k_j\in\N,\, k_1,\ldots,k_j\ge 1,\, k_1+\cdots+k_j=i} a_j b_{k_1} \cdots b_{k_j} \right) X^i
    (b \circ a)(X) := \sum_{i\in\N} \left(\sum_{k\in\N,\, j_1,\ldots,j_k\in\N,\, j_1,\ldots,j_k\ge 1,\, j_1+\cdots+j_k=i} b_k a_{j_1} \cdots a_{j_k} \right) X^i
  \]
  で定める.
  内側の $\sum$ について,$k \le i$ が従うためこれは有限和である.
  特に,$(b \circ a)(X)$ の定数項は $b_0$ である.
\end{dfn}

$4$ 次以下の項を書き下すと
\begin{align*}
  (b \circ a)(X)
  = b_0
  + b_1 a_1 X
  + (b_1 a_2 + b_2 a_1^2) X^2
  + (b_1 a_3 + 2 b_2 a_1 a_2 + b_3 a_1^3) X^3 \\
  + (b_1 a_4 + b_2 (a_1 a_3 + a_2^2) + 3 b_3 a_1^2 a_2 + b_4 a_1^4) X^4
  + \cdots
\end{align*}
となる.

$(b \circ a)(X)$ の $i$ 次の係数は,$b_k a(X)^k$ の $i$ 次の係数を $k \in \N$ について足したものである.
つまり,形式的に $(b \circ a)(X) = \sum_{k\in\N} b_k a(X)^k$ と書きたいが,
右辺は $R[[X]]$ の元の無限和であり定義されておらず,各係数ごとに有限和として定義できるための条件が $a(X)$ の定数項が $0$ であることに他ならない.
このとき,$k > i$ の項は $i$ 次の係数に影響を与えない.
言い換えると,
\begin{prp}
  \label{prp:composition-mod}
  $a(X) = \sum_{i\in\N,\,i\ge 1} a_i X^i \in X R[[X]]$ と
  $b(X) = \sum_{i\in\N} b_i X^i \in R[[X]]$ に対し,
  \[
    (b \circ a)(X) \equiv \sum_{k\in\N,\,k<n} b_k a(X)^k \pmod{X^n}
  \]
  が成り立つ.
\end{prp}
\begin{prf}
  $i \in \N$, $i < n$ に対し,$k \le i$ ならば $k < n$ であるから,
  \begin{align*}
    % (a \circ b)(X)
    % &\equiv \sum_{i\in\N,\,i<n} \left(\sum_{j\in\N,\, k_1,\ldots,k_j\in\N,\, k_1,\ldots,k_j\ge 1,\, k_1+\cdots+k_j=i} a_j b_{k_1} \cdots b_{k_j} \right) X^i \\
    % &= \sum_{i\in\N,\,i<n} \left(\sum_{j\in\N,\, j<n,\, k_1,\ldots,k_j\in\N,\, k_1,\ldots,k_j\ge 1,\, k_1+\cdots+k_j=i} a_j b_{k_1} \cdots b_{k_j} \right) X^i \\
    % &= \sum_{i\in\N,\,i<n} \left( \sum_{j\in\N,\,j<n} [X^i] a_j b(X)^j \right) X^i \\
    % &= \sum_{j\in\N,\,j<n} \left( \sum_{i\in\N,\,i<n} [X^i] a_j b(X)^j \right) X^i \\
    % &\equiv \sum_{j\in\N,\,j<n} a_j b(X)^j \pmod{X^n}
    (b \circ a)(X)
    &\equiv \sum_{i\in\N,\,i<n} \left(\sum_{k\in\N,\, j_1,\ldots,j_k\in\N,\, j_1,\ldots,j_k\ge 1,\, j_1+\cdots+j_k=i} b_k a_{j_1} \cdots a_{j_k} \right) X^i \\
    &= \sum_{i\in\N,\,i<n} \left(\sum_{k\in\N,\, k<n,\, j_1,\ldots,j_k\in\N,\, j_1,\ldots,j_k\ge 1,\, j_1+\cdots+j_k=i} b_k a_{j_1} \cdots a_{j_k} \right) X^i \\
    &= \sum_{i\in\N,\,i<n} \left( \sum_{k\in\N,\,k<n} [X^i] b_k a(X)^k \right) X^i \\
    &= \sum_{k\in\N,\,k<n} \left( \sum_{i\in\N,\,i<n} [X^i] b_k a(X)^k \right) X^i \\
    &\equiv \sum_{k\in\N,\,k<n} b_k a(X)^k \pmod{X^n}
  \end{align*}
  である.
\end{prf}

とくに,$a(X) = X$ ならば $(b \circ a)(X) = b(X)$ であることが命題 \ref{prp:mod-limit} よりわかる.

合成を「代入」と考えたとき成り立ってほしい性質たちをさらに確認していく.

\begin{prp}
  \label{prp:composition-hom}
  $a(X) \in X R[[X]]$ は $R$ 代数の準同型 $a^*\colon R[[X]] \longto R[[X]]$; $b(X) \longmapsto (b \circ a)(X)$ を定め,
  これは $a^*(X) = a(X)$ を満たす.
  すなわち,$b(X), c(X), d(X) \in R[[X]]$ に対し,
  \begin{enumerate}[(1)]
    \item $b(X) + c(X) = d(X)$ ならば $(b \circ a)(X) + (c \circ a)(X) = (d \circ a)(X)$.
    \item $b(X) c(X) = d(X)$ ならば $(b \circ a)(X) (c \circ a)(X) = (d \circ a)(X)$.
    \item $b(X) = 1$ ならば $(b \circ a)(X) = 1$.
    \item $b(X) = X$ ならば $(b \circ a)(X) = a(X)$.
  \end{enumerate}
\end{prp}

\begin{prf}
  \begin{enumerate}[(1)]
    \item 合成の定義から明らか.
    \item $a(X) = \sum_{i\in\N,\,i\ge 1} a_i X^i$ および $b(X) = \sum_{i\in\N} b_i X^i$, $c(X) = \sum_{i\in\N} c_i X^i$ とする.
        $n \in \N$ を任意にとる.
        命題 \ref{prp:composition-mod} より,
        \begin{align*}
          % (a \circ d)(X) (b \circ d)(X)
          % &\equiv \left( \sum_{i\in\N,\,i<n} a_i d(X)^i \right) \left( \sum_{j\in\N,\,j<n} b_j d(X)^j \right) \\
          % &= \sum_{k\in\N,k<2n} \left( \sum_{i,j\in\N,\,i,j<n,\,i+j=k} a_i b_j \right) d(X)^k \\
          % &\equiv \sum_{k\in\N,k<n} \left( \sum_{i,j\in\N,\,i+j=k} a_i b_j \right) d(X)^k \\
          % &= \sum_{k\in\N,k<n} c_k d(X)^k \\
          % &\equiv (c \circ d)(X) \pmod{X^n}
          (b \circ a)(X) (c \circ a)(X)
          &\equiv \left( \sum_{i\in\N,\,i<n} b_i a(X)^i \right) \left( \sum_{j\in\N,\,j<n} c_j a(X)^j \right) \\
          &= \sum_{k\in\N,k<2n} \left( \sum_{i,j\in\N,\,i,j<n,\,i+j=k} b_i c_j \right) a(X)^k \\
          &\equiv \sum_{k\in\N,k<n} \left( \sum_{i,j\in\N,\,i+j=k} b_i c_j \right) a(X)^k \\
          &= \sum_{k\in\N,k<n} d_k a(X)^k \\
          &\equiv (d \circ a)(X) \pmod{X^n}
        \end{align*}
        である.
        よって,命題 \ref{prp:mod-limit} より $(a \circ d)(X) (b \circ d)(X) = (c \circ d)(X)$ が従う.
    \item $b_0$ のみ $1$ なので,$(b \circ a)(X) = \sum_{i\in\N} \left( \sum_{0=i} 1 \right) X^i = 1$.
    \item $b_1$ のみ $1$ なので,$(b \circ a)(X) = \sum_{i\in\N} \left( \sum_{k_1\in\N,\,k_1\ge 1,\,k_1=i} a_{k_1} \right) X^i = \sum_{i\in\N} a_i X^i = a(X)$.
  \end{enumerate}
\end{prf}

\begin{prp}
  $a(X), b(X) \in X R[[X]]$ と $c(X) \in R[[X]]$ に対し,
  $(c \circ (b \circ a))(X) = ((c \circ b) \circ a)(X)$.
\end{prp}
\begin{prf}
  $c(X) = X$ のとき,命題 \ref{prp:composition-hom} (4) より,
  \[
    (c \circ (b \circ a))(X) = (b \circ a)(X) = ((c \circ b) \circ a)(X)
  \]
  である.
  すなわち $(b \circ a)^*(X) = (a^* \circ b^*)(X)$ である ($(b \circ a)(X)$ は形式的冪級数の合成,$a^* \circ b^*$ は $R$ 代数の準同型の合成であることに注意する).

  TODO ↓嘘!!!生成されてない.位相を入れて連続準同型なら決まる

  $R[[X]]$ は $X$ で生成されるので,準同型は $X$ の行き先で定まる.
  よって $(b \circ a)^* = a^* \circ b^*$ であり,
  任意の $c(X)$ に対し
  \[
    (c \circ (b \circ a))(X) = (b \circ a)^*(c(X)) = (a^* \circ b^*)(c(X)) = ((c \circ b) \circ a)(X)
  \]
  となる.
\end{prf}

これらの理解のもと,$(b \circ a)(X)$ を $b(a(X))$ とも書く.
とくに,$b(0)$ は $b(X)$ の定数項に等しい.

\begin{exm}
  $a(X) = \sum_{i\in\N} a_i X^i \in R[[X]]$ と正の整数 $n$ に対し,
  $a(X^n) = \sum_{i\in\N} a_i X^{ni}$.
\end{exm}

\begin{exm}
  $a(X) = \frac{X}{1 - X} = \sum_{i\in\N} X^{i+1} \in R[[X]]$ と $n \in \N$ に対し,
  $(\underbrace{a \circ \cdots \circ a}_{n})(X) = \frac{X}{1 - n X} = \sum_{i\in\N} n^i X^{i+1}$
  ($n$ 回合成を $a^n$ と書くと $a^n(X)$ か $a(X)^n$ かかなり紛らわしいため避けている).
\end{exm}


\section{形式微分}
多項式の微分を拡張して形式微分が定義できる.
記法についてはいくつかの流儀・用途がある.

\begin{dfn}
  $R$ 加群の準同型 $D\colon R((X)) \longto R((X))$ を,
  $a(X) = \sum_{i\in\Z} a_i X^i \in R((X))$ に対し,
  \[
    D(a(X)) = \sum_{i\in\Z} i a_i X^{i-1}
  \]
  として定める.
  $D(a(X))$ を $a'(X)$ とも書く.
  $D$ を ($X$ による) \textbf{形式微分}と呼ぶ.
\end{dfn}

$0 a_0 X^{-1} = 0$ なので,$D$ を $R[[X]]$ に制限すると
$R$ 加群の準同型 $D\colon R[[X]] \longto R[[X]]$ が得られる.

$D$ は $R$ 加群としては準同型である (線型性を満たす) が
$R$ 代数の準同型ではない (積は保たない) ことに注意する.
積に関しては,以下のいわゆる Leibniz rule を満たす:

\begin{prp}
  \label{prp:leibniz}
  $a(X), b(X) \in R((X))$ に対し,$D(a(X) b(X)) = D(a(X)) b(X) + a(X) D(b(X))$.
\end{prp}
\begin{prf}
  $a(X) = \sum_{i\in\Z,\,i\ge m} a_i X^i$, $b(X) = \sum_{i\in\Z,\,i\ge n} b_i X^i$ ($m, n \in \Z$) とすると,
  \begin{align*}
    D(a(X) b(X))
    &= D\left(\sum_{k\in\Z,\,k\ge m+n} \left(\sum_{i,j\in\Z,\,i\ge m,\,j\ge n,\,i+j=k} a_i b_j\right) X^k\right) \\
    &= \sum_{k\in\Z,\,k\ge m+n} k \left(\sum_{i,j\in\Z,\,i\ge m,\,j\ge n,\,i+j=k} a_i b_j\right) X^{k-1} \\
    &= \sum_{k\in\Z,\,k\ge m+n} \left(\sum_{i,j\in\Z,\,i\ge m,\,j\ge n,\,i+j=k} (i a_i b_j + j a_i b_j)\right) X^{k-1} \\
    &= \left(\sum_{i\in\Z,\,i\ge m} i a_i X^{i-1}\right) \left(\sum_{j\in\Z,\,j\ge n} b_j X^j\right)
        + \left(\sum_{i\in\Z,\,i\ge m} a_i X^i\right) \left(\sum_{j\in\Z,\,j\ge n} j b_j X^{j-1}\right) \\
    &= D(a(X)) b(X) + a(X) D(b(X))
  \end{align*}
  より成り立つ.
\end{prf}

合成に関しては,以下のいわゆる chain rule を満たす\footnote{$b'(a(X))$ が $D(b(X))$ と $a(X)$ の合成であることに注意 (記法のせいで $D$ で綺麗に書けない).}:

\begin{prp}
  $a(X) \in X R[[X]]$ と $b(X) \in R[[X]]$ に対し,
  $D((b \circ a)(X)) = b'(a(X)) a'(X)$.
\end{prp}
\begin{prf}
  $k \in \N$ に対し,$D(a(X)^k) = k a(X)^{k-1} a'(X)$ である.
  これは,$k = 0$ のときはよく,$k \ge 1$ のときは命題 \ref{prp:leibniz} を $k - 1$ 回用いる.

  $n \in \N$ を任意にとる.
  命題 \ref{prp:composition-mod} より,
  $(b \circ a)(X) \equiv \sum_{k\in\N,\,k<n} b_k a(X)^k \pmod{X^{n+1}}$ なので,
  \begin{align*}
    D((b \circ a)(X))
    &\equiv D\left(\sum_{k\in\N,\,k<n} b_k a(X)^k\right) \\
    &= \sum_{k\in\N,\,k<n} b_k D(a(X)^k) \\
    &= \sum_{k\in\N,\,k<n} b_k \cdot k a(X)^{k-1} a'(X) \\
    &= \left(\sum_{k\in\N,\,k<n} k b_k a(X)^{k-1}\right) a'(X)
    \pmod{X^n}
  \end{align*}
  となる.
  よって命題 \ref{prp:mod-limit} より $D((b \circ a)(X)) = b'(a(X)) a'(X)$ が従う.
\end{prf}

\begin{exm}
  $a(X) = \sum_{i\in\Z} a_i X^i \in R((X))$ に対し,$a'(0) = a_1$ である.
  より一般に,$n$ 階微分 ($n \in \Z$) を考えると,
  $a^{(n)}(X) = \underbrace{D(\cdots D}_n (a(X)) \cdots)$ として $a^{(n)}(0) = n! a_n$ である.
\end{exm}

\begin{exm}
  $(1 - X)^{-1} = \sum_{i\in\N} X^i$ について,
  $D((1 - X)^{-1}) = \sum_{i\in\N,\,i\ge 1} i X^{i-1} = (1 - X)^{-2}$.
\end{exm}


\section{形式積分}
この節では $K$ を標数 $0$ の体とする.

微分の「逆操作」として積分を考えることができる.

\begin{dfn}
  $a(X) = \sum_{i\in\Z} a_i X^i \in K((X))$ が $a_{-1} = 0$ を満たすとき,
  \[
    I(a(X)) = \sum_{i\in\Z,\,i\ne -1} \frac{1}{i + 1} a_i X^{i+1}
  \]
  と定める.
  $I(a(X))$ を $\int a(X) dX$ とも書く.
  $I$ を ($X$ による) \textbf{形式積分}と呼ぶ.
\end{dfn}

$I$ は部分 $K$ 加群間の準同型
$\{ a(X) \in K((X)) \mid [X^{-1}] a(X) = 0 \} \longto \{ a(X) \in K((X)) \mid [X^0] a(X) = 0 \}$ を与える.
また,$I$ を $K[[X]]$ に制限すると
$K$ 加群の準同型 $I\colon K[[X]] \longto X K[[X]]$ が得られる.

\begin{prp}
  $a(X) \in K((X))$ に対し,
  \begin{enumerate}[(1)]
    \item $I(D(a(X))) = a(X) - a(0)$.
    \item $[X^{-1}] a(X) = 0$ ならば,$D(I(a(X))) = a(X)$.
  \end{enumerate}
\end{prp}
\begin{prf}
  定義より明らか.
\end{prf}

\section{形式留数}
微分で情報が落ちる部分に名前がついている.

\begin{dfn}
  $a(X) \in R((X))$ に対し,$[X^{-1}] a(X)$ を $\Res(a(X))$ とも書き,
  $a(X)$ の\textbf{形式留数}という.
\end{dfn}

$\Res\colon R((X)) \longto R$ は $R$ 準同型である.

\begin{prp}
  $a(X), b(X) \in X R[[X]]$ が $b(a(X)) = X$, $a(b(X)) = X$ を満たすとき,
  $m, n \in \N$ に対し,
  \[
    m [X^m] b(X)^n = n [X^{-n}] a(X)^{-m}
  \]
  が成り立つ.
\end{prp}
\begin{prf}
  TODO: 体じゃなくて大丈夫?
\end{prf}


\section{exp}
この節では $K$ を標数 $0$ の体とする.

\begin{dfn}
  $\exp(X) \in R[[X]]$ を,
  \[
    \exp(X) = \sum_{i\in\N} \frac{1}{i!} X^i
  \]
  で定める.
\end{dfn}

定義から,$D(\exp(X)) = \exp(X)$ がわかる.

$\exp$ を左から合成する写像 $\exp\colon X K[[X]] \longto 1 + X K[[X]]; a(X) \longmapsto \exp(a(X))$ は指数法則を満たす:

\begin{prp}
  $a(X), b(X) \in X K[[X]]$ に対し,$\exp(a(X) + b(X)) = \exp(a(X)) \exp(b(X))$.
\end{prp}
\begin{prf}
  $a(X) = \sum_{i\in\N,\,i\ge 1} a_i X^i$, $b(X) = \sum_{i\in\N,\,i\ge 1} b_i X^i$ とする.
  $n \in \N$ を任意にとる.
  命題 \ref{prp:composition-mod} より,
  \begin{align*}
    \exp(a(X) + b(X))
    &\equiv \sum_{k\in\N,k<n} \frac{1}{k!} (a(X) + b(X))^k \\
    &= \sum_{k\in\N,k<n} \sum_{i,j\in\N,\,i+j=k} \frac{1}{i! j!} a(X)^i b(X)^j \\
    &\equiv \left(\sum_{i\in\N,i<n} \frac{1}{i!} a(X)^i\right) \left(\sum_{j\in\N,j<n} \frac{1}{j!} b(X)^j\right) \\
    &\equiv \exp(a(X)) \exp(b(X))
    \pmod{X^n}
  \end{align*}
  となる ($2$ つ目の $\equiv$ は $a(X)^i \equiv 0 \pmod{X^i}$, $b(X)^j \equiv 0 \pmod{X^j}$ を用いた).
  よって命題 \ref{prp:mod-limit} より $\exp(a(X) + b(X)) = \exp(a(X)) \exp(b(X))$ が従う.
\end{prf}




\section{log}
\section{合成逆}
\section{多変数}
\section{アルゴリズム}
\section{数え上げ}


\end{document}
